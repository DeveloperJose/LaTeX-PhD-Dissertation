% abstract.tex (Abstract)

\addcontentsline{toc}{chapter}{Abstract}

% The abstract should be approximately 200-400 words
\chapter*{Abstract}
Deep neural network models are state-of-the-art for many image, audio, text, and video processing problems in different fields of study and disciplines. However, training many of these networks can require a lot of data and gathering such data can be time consuming, costly, and difficult to set-up.  This limiting factor can prevent researchers and engineers that do not have access to a lot of resources from using all the tools and models available in deep learning to tackle novel problems in their respective fields. To work around these data limitations, many techniques and models in the field of Few-Shot Learning have been proposed over the years in the sub-fields of transfer learning, data augmentation, and meta learning. One such model, Physics-Informed Neural Networks or PINNs, relies on the fact that datasets collected in the real world must follow the laws of physics, allowing us to leverage our knowledge of physics to help improve performance on these datasets without requiring more data to be gathered. I propose 4 ways of extending this type of approach that leverages our knowledge of physics through new models and data augmentation techniques in this dissertation, and apply these approaches to two problems that have limited datasets: the problem of Fluid Flow Velocity Prediction in the field of mechanical engineering and the problem of Glacial Ice Segmentation from geology. 

A detailed explanation of the importance of data gathering, difficulties and limitations, and my expected contributions for this area along with a timeline are presented in Chapter 1. The remaining chapters each describe one of the four expected contributions of my dissertation in chronological order.