\chapter{Physics-Informed Network For Glacial Ice Velocity Predictions}
As my next step, I propose making a new Physics-Informed architecture based on what I learned from the Physics-Informed LSTM Network \cite{Perez2022} which I previously applied to fluid flow simulations and adapting it to the problem of glacial ice velocity prediction. Glacial ice velocity prediction is simply the problem of fluid flow velocity prediction where the fluid is ice from a glacier. The aim of this step is to prepare a network architecture and dataset than can be later incorporated into the glacier segmentation problem as I hypothesize that including velocity information based on the physics laws of fluid flow will improve the performance of the glacier segmentation models proposed by \cite{Bibek2023}. There already exists a dataset of glacier ice velocities \cite{GoLIVE1} for satellite images created by the National Snow and Ice Data Center which can be easily adapted and incorporated with my previous methodology for velocity predictions. However, since then I have learned about newer architectures that perform better than LSTMs \cite{LSTM} such as GRUs \cite{GRU} and Transformers \cite{VIT} which I would like to try as well.

The main challenge with this step is determining how to set-up the Physics-Informed part of the network as my previous work focused on air and water flows and not ice flows. The architecture will not require many changes, it is the loss function which was based on the incompressible 2D Navier-Stokes equations which might need to be modified to accommodate the new type of data.